% !TeX engine = xelatex
% !TeX spellcheck = en-US
% adjust to 16:9 for displays. you can just use \documentclass[handout]{ctexbeamer} for notes

\documentclass[aspectratio=169]{beamer}
\usepackage{booktabs}
\usepackage{svg}
\usepackage{fontawesome}
\usepackage[style=authortitle-comp,backend=bibtex]{biblatex}
\usecolortheme{seagull}
\setbeamertemplate{sidebar right}{}
\setbeamertemplate{footline}{%
	\hfill\usebeamertemplate***{navigation symbols}
    \hspace{1cm}\insertframenumber{}/\inserttotalframenumber}

\title{Understanding, Detecting and Localizing Partial Failures in Large System Software\footfullcite{lou2020understanding}}
\date{\today}

\addbibresource{ref.bib}

\begin{document}

\begin{frame}
	\titlepage
\end{frame}

\begin{frame}{Overview}
    \tableofcontents
  \end{frame}

\section{Problem definition}

\begin{frame}
    \frametitle{What is a Partial Failure?}
    \begin{definition}
        A partial failure is, in a process $\pi$ to be when a fault \textbf{does not} crash $\pi$ but causes safety or liveness violation or severe slowness for some functionality $R_f \subsetneq  R$

    \end{definition}
\end{frame}
    
\end{document}